\documentclass[10pt, compress]{beamer}

\usetheme{m}
\usepackage{textcomp}
\usepackage{booktabs}
\usepackage[scale=2]{ccicons}
\usepackage{minted}
\usepackage{smartdiagram}

\usemintedstyle{trac}

\title{Data Management and Introduction to Stata}
\subtitle{}
\date{Thursday 12th April}
\author{David Clark}
\institute{University of Leeds}

\begin{document}

\maketitle

\section{Introductions and Preamble}

\begin{frame}[fragile]
  \frametitle{Introductions}

\textbf{David Clark}
\begin{itemize}
    \item Teaching Fellow in Economics
    
    \item Used Stata for 6(-ish) years ... still learning!
\end{itemize}
    
   \textbf{Kausik Chaudhuri}
\begin{itemize}
    \item  Senior Lecturer in Economics
    
   \item  Used Stata for ..... 
    
\end{itemize}

\end{frame}

\begin{frame}[fragile]
  \frametitle{Who is this course for?}
    Targeted at anyone who has \textbf{no to little} experience using Stata
    
    Primarily for those engaging in \alert{quantitative} research (MRes/PhD)
    
    What to learn to use a statistical package that allows for both use of \textbf{point-and-click GUI} and \textbf{Stata's Markup and Control Language (SMCL)}
      
\end{frame}

\begin{frame}[fragile]
  \frametitle{Who is this course for?}

    For those who want to:
        \begin{itemize}
            \item \textbf{Organise and manage data}
                \begin{itemize}
                    \item Generating, reshaping, dropping and recoding
                \end{itemize}
            \item \textbf{Visualise data}
                \begin{itemize}
                    \item Scatter and line graphs
                    \item Histograms
                \end{itemize}
            \item \textbf{Analyse data}
                \begin{itemize}
                    \item ANOVA
                    \item Regression analysis
                \end{itemize}
            \item \textbf{Automate and reproduce workflow}
                \begin{itemize}
                    \item Log and do-files
                    \item Loops
                \end{itemize}
        \end{itemize}
      
\end{frame}

\section{What is Stata?}

\begin{frame}[fragile]
  \frametitle{What is Stata?}
  Stata is a powerful statistical package with:
  \begin{itemize}

   \item  smart data-management facilities
  
   \item  a wide array of up-to-date statistical techniques
  
   \item  an excellent system for producing publication-quality graphs
    
  \end{itemize}
  
  Available on a variety of operating systems (Windows, Mac OS and Linux distributions)
  
  Also available in different varieties:
  
    \begin{itemize}

   \item  IC (standard)
  
   \item SE (extended)
  
   \item  MP (multiprocessing)
    
  \end{itemize}
  
\end{frame}


\begin{frame}[fragile]
\frametitle{Why not use X?}
There are alternative statistical software packages you can use (to name a few):

\begin{itemize}
    \item R
    \item Matlab
    \item SAS
    \item SPSS
    \item Gauss
    \item Gretl
    \item Eviews
\end{itemize}

\end{frame}

\begin{frame}[fragile]
  \frametitle{Stata 14 Front End Graphic User Interface (GUI)}
  \includegraphics[width=\linewidth,height=\textheight,keepaspectratio]{stata.png}
\end{frame}

\begin{frame}[fragile]
  \frametitle{Stata 14 Front End GUI}
Stata has an menu bar on the top and 5 internal windows.

The \alert{main} window is the one in the middle (1 on the previous slide). It gives
you all the output of your operations in Stata.

The \alert{command window} (2) executes commands. 
\begin{itemize}
    \item You can type commands directly in this window as an alternative to using the menu system.
    \item Stata will show you what the written command is for each action performed using the drop-down menus.
\end{itemize}


\end{frame}

\begin{frame}[fragile]
  \frametitle{Stata 14 Front End GUI}
  
  The \alert{review window} (3), lists all the operations preformed since opening
Stata. If you click on one of your past commands, you will see the command
being displayed in the Command window and you can re-run it by hitting the
enter key.

The \alert{variables window} (4) lists the variables in the current dataset (and their
descriptions). When you double-click on the variable, it appears in the
Command window.

The \alert{properties window} (5) gives information about your dataset and your
variables.

\end{frame}

\section{Stata Workflow}

\begin{frame}[fragile]
\frametitle{Stata Workflow}
\begin{center}
\usetikzlibrary{shapes.geometric} % required in the preamble
\smartdiagramset{module shape=rectangle,
font=\tiny,
module minimum width=2cm,
module minimum height=1cm,
text width=1.7cm,
circular distance=2cm,
circular final arrow disabled=true,
}
\smartdiagram[circular diagram:clockwise]{File management, Data management, Descriptive statistics, Visualisation, Statistical analysis}
\end{center}	
\end{frame}

\newcommand{\insm}{%
\smash{\raisebox{.5\dimexpr3\baselineskip+4\itemsep+2\parskip}{$\left.\rule{0pt}{.5\dimexpr2\baselineskip+2\itemsep+2\parskip}\right\}$\ \parbox{5.5cm}{These two stages will consume the \alert{most time} in any research project}}}
}


\begin{frame}[fragile]
\frametitle{Stata Workflow}
\begin{itemize}
\item \textbf{File management}
\item \textbf{Data management}
\item \textbf{Descriptive statistics} \insm
\item \textbf{Visualisation}
\item \textbf{Statistical analysis}
\end{itemize}
\end{frame}

\section{File Management}

\begin{frame}[fragile]
\frametitle{Stata Workflow: File Management}
\begin{center}
\usetikzlibrary{shapes.geometric} % required in the preamble
\smartdiagramset{module shape=rectangle,
font=\tiny,
module minimum width=2cm,
module minimum height=1cm,
text width=1.7cm,
circular distance=2cm,
circular final arrow disabled=true,
}
\smartdiagram[circular diagram:clockwise]{\textbf{File management}, Data management, Descriptive statistics, Visualisation, Statistical analysis}
\end{center}	
\end{frame}

\begin{frame}[fragile]
\frametitle{File Management}
\begin{itemize}
\item This is often an aspect of using Stata that is \textbf{wrongly} overlooked
\item Usually a facet that people return to after learning the syntax
\item As researchers, \alert{one of our primary objectives}:
\end{itemize}
\begin{center}
	\textbf{Replicability and reliability}
\end{center}
\begin{itemize}
\item If, after testing your research hypothesis, using data, you discover some results of interest, \textbf{what use is this if they cannot be reproduced by others?}
\item Hence, engraining good practices from the beginning, \alert{promotes higher-quality research} in future work
\end{itemize}
\end{frame}

\begin{frame}[fragile]
\frametitle{File Management}
\begin{itemize}
\item What do we mean by \textbf{file management}?
\begin{itemize}
\item Typically, when people (\alert{most}) begin using Stata, they will just open some data and \textbf{do stuff}
\end{itemize}
\item Questions that arise:
\begin{itemize}
\item \textbf{Where is the data stored?}
\item \textbf{Where is the output stored?}
\item \textbf{Where is Stata currently working from?}
\item \textbf{Are we utilising one or many directories?}
\end{itemize}
\item File management is knowing the answer to these questions constantly and having a good justification for their placement
\end{itemize}
\end{frame}

\begin{frame}[fragile]
\frametitle{File Management}
\begin{center}
	\textbf{Where is Stata currently working from?}
\end{center}
\begin{itemize}
\item \alert{Definition}: \textbf{working directory}
\begin{itemize}
\item The (\alert{current}) working directory is the file within the computer's hierarchical file structure that a program is working from
\end{itemize}
\item That is to say, anything you ask Stata to open or to save will be accessed or stored in this working directory
\end{itemize}
\end{frame}

\begin{frame}[fragile]
\frametitle{File Management}
\begin{center}
	\textbf{Where is Stata currently working from?}
\end{center}
\begin{itemize}
\item There are two ways of finding out what the current working directory is in Stata:
\begin{itemize}
\item Look at the bottom-left hand corner of Stata
\end{itemize}
\smallskip
\begin{center}
\includegraphics{pwd1.png}
\end{center}
\begin{itemize}
\item Type the command \textbf{pwd} into the command window in Stata
\end{itemize}
\begin{center}
\includegraphics{pwd2.png}
\end{center}
\item Both are telling us that we are working out of the \textbf{Downloads} folder
\end{itemize}
\end{frame}

\begin{frame}[fragile]
\frametitle{File Management}
\begin{itemize}
\item On the University system, this usually is set as a default to the personal drive (M:/)
\item In either case, \textbf{is it a good idea to work out of an indiscriminate folder?}
\begin{itemize}
\item \textbf{Almost always, no!}
\item Why? $\to$ There will be unrelated files that will make it complex to keep track of related files and output
\end{itemize}
So, we have two options what we can proceed with that adhere to \textbf{good practice}:
\begin{itemize}
\item \alert{Change} to a directory that already exists
\item \alert{Create} a directory to work from
\end{itemize}
\end{itemize}
\end{frame}

\begin{frame}[fragile]
\frametitle{File Management: Changing Directory}
\begin{itemize}
\item If the folder that you want to work from \alert{already exists}, we can tell Stata to change the working directory to this folder.
\item For example, imagine I have a folder called \textbf{Thesis$\_$Paper$\_$One} and here is the path (note, this was the file path on my Mac, it will look slightly different on Windows PCs):
\end{itemize}
\begin{center}
\textbf{Users $\to$ David $\to$ Documents $\to$ Projects $\to$ Thesis$\_$Paper$\_$One}
\end{center}
\begin{itemize}
\item This can be done in two ways:
\begin{itemize}
\item Using the drop down menus in the GUI
\item Using the \textbf{cwd} command directly
\end{itemize}
\end{itemize}
\end{frame}

\begin{frame}[fragile]
\frametitle{File Management: Changing Directory}
\textbf{Using the drop down menus in the GUI}
\begin{itemize}
\item If you follow this menu path:
\end{itemize}
\begin{center}
\textbf{File $\to$ Change working directory...}
\end{center}
\begin{itemize}
\item Stata will then open a \alert{file explorer window} where you can navigate to, and choose, the folder you wish to set as the current working directory
\item This is a useful method if \textbf{you do not have the exact file path to hand}
\item Notice, Stata will then print the exact file path in the output window after changing working directory successfully.
\end{itemize}
\end{frame}

\begin{frame}[fragile]
\frametitle{File Management: Changing Directory}
\textbf{Using the drop down menus in the GUI}
\begin{itemize}
\item If you already happen to know the file path to the directory, we can type the change directory command directly into the command prompt:
\end{itemize}
\begin{center}
\mint[fontfamily=tt]{stata}| cd "/Users/David/Documents/Projects/Thesis_Paper_One" |
\end{center}
\textbf{Breakdown}
\begin{itemize}
\item \mint[fontfamily=tt]{stata}|cd| Tells Stata to \textbf{change directory}
\item \mint[fontfamily=tt]{stata}| "/Users/David/Documents/Projects/Thesis_Paper_One" | \textbf{Provides Stata with the file path} to the directory that you will want to change to
\end{itemize}
\end{frame}

\begin{frame}[fragile]
\frametitle{File Management: Creating a Directory}
\begin{itemize}
\item Perhaps you want to create the folder, as part of a new project, which we'll call \textbf{Thesis$\_$Paper$\_$Two}
\item Here, we can only use the command prompt, by typing the following command
\end{itemize}
\begin{center}
\mint[fontfamily=tt, fontsize=\small]{stata}| mkdir "/Users/David/Documents/Projects/Thesis_Paper_Two" |
\end{center}
\textbf{Breakdown}
\begin{itemize}
\item \mint[fontfamily=tt, fontsize=\small]{stata}|mkdir| Tells Stata to \textbf{create a new folder in this directory}
\item \mint[fontfamily=tt, fontsize=\small]{stata}| "/Users/David/Documents/Projects/Thesis_Paper_Two" | \textbf{Provides Stata with the file path} to the directory that you will want to move to (\alert{Projects}) and create a folder in there called \alert{Thesis$\_$Paper$\_$Two}
\end{itemize}
\end{frame}

\section{Demo: Changing and Creating Directories}

\section{Data Management}

\begin{frame}[fragile]
\frametitle{Stata Workflow: Data Management}
\begin{center}
\usetikzlibrary{shapes.geometric} % required in the preamble
\smartdiagramset{module shape=rectangle,
font=\tiny,
module minimum width=2cm,
module minimum height=1cm,
text width=1.7cm,
circular distance=2cm,
circular final arrow disabled=true,
}
\smartdiagram[circular diagram:clockwise]{File management, \textbf{Data management}, Descriptive statistics, Visualisation, Statistical analysis}
\end{center}	
\end{frame}


\begin{frame}[fragile]
\frametitle{Data Management}
\begin{itemize}
\item As stated previously, the data management aspect of the workflow is arguably \textbf{one of the most important (and time-consuming) stages of a research project}
\item \textbf{Why?}
\begin{itemize}
\item Data \alert{might not be native to Stata}, so it must be imported correctly
\item Datasets, particularly survey data, may have some \alert{errors in their reporting and may require our attention}
\item You may want to gather data from different datasets and \alert{consolidate them into one master dataset}
\item Perhaps you want to \alert{create new variables} based on the original data
\end{itemize}
\item Taking the time to carry out this stage properly will \textbf{save you time in the long run}
\end{itemize}
\end{frame}

\begin{frame}[fragile]
  \frametitle{Typography}
      \begin{minted}[fontsize=\small]{latex}
The theme provides sensible defaults to \emph{emphasis}
text, \alert{accent} parts or show \textbf{bold} results.
      \end{minted}

  \begin{center}becomes\end{center}

  The theme provides sensible defaults to \emph{emphasis} text,
  \alert{accent} parts or show \textbf{bold} results.
\end{frame}
\begin{frame}{Lists}
  \begin{columns}[onlytextwidth]
    \column{0.5\textwidth}
      Items
      \begin{itemize}
        \item Milk \item Eggs \item Potatos
      \end{itemize}

    \column{0.5\textwidth}
      Enumerations
      \begin{enumerate}
        \item First, \item Second and \item Last.
      \end{enumerate}
  \end{columns}
\end{frame}
\begin{frame}{Descriptions}
  \begin{description}
    \item[PowerPoint] Meeh.
    \item[Beamer] Yeeeha.
  \end{description}
\end{frame}
\begin{frame}{Animation}
  \begin{itemize}[<+- | alert@+>]
    \item \alert<4>{This is\only<4>{ really} important}
    \item Now this
    \item And now this
  \end{itemize}
\end{frame}
\begin{frame}{Figures}
  \begin{figure}
    \newcounter{density}
    \setcounter{density}{20}
    \begin{tikzpicture}
      \def\couleur{mLightBrown}
      \path[coordinate] (0,0)  coordinate(A)
                  ++( 90:5cm) coordinate(B)
                  ++(0:5cm) coordinate(C)
                  ++(-90:5cm) coordinate(D);
      \draw[fill=\couleur!\thedensity] (A) -- (B) -- (C) --(D) -- cycle;
      \foreach \x in {1,...,40}{%
          \pgfmathsetcounter{density}{\thedensity+20}
          \setcounter{density}{\thedensity}
          \path[coordinate] coordinate(X) at (A){};
          \path[coordinate] (A) -- (B) coordinate[pos=.10](A)
                              -- (C) coordinate[pos=.10](B)
                              -- (D) coordinate[pos=.10](C)
                              -- (X) coordinate[pos=.10](D);
          \draw[fill=\couleur!\thedensity] (A)--(B)--(C)-- (D) -- cycle;
      }
    \end{tikzpicture}
    \caption{Rotated square from
    \href{http://www.texample.net/tikz/examples/rotated-polygons/}{texample.net}.}
  \end{figure}
\end{frame}
\begin{frame}{Tables}
  \begin{table}
    \caption{Largest cities in the world (source: Wikipedia)}
    \begin{tabular}{lr}
      \toprule
      City & Population\\
      \midrule
      Mexico City & 20,116,842\\
      Shanghai & 19,210,000\\
      Peking & 15,796,450\\
      Istanbul & 14,160,467\\
      \bottomrule
    \end{tabular}
  \end{table}
\end{frame}
\begin{frame}{Blocks}

  \begin{block}{This is a block title}
    This is soothing.
  \end{block}

\end{frame}
\begin{frame}{Math}
  \begin{equation*}
    e = \lim_{n\to \infty} \left(1 + \frac{1}{n}\right)^n
  \end{equation*}
\end{frame}
\begin{frame}{Quotes}
  \begin{quote}
    Veni, Vidi, Vici
  \end{quote}
\end{frame}

\plain{Dark background}{\vspace{-2em}\begin{center}\includegraphics[width=\textwidth]{images/valley.jpg}\end{center}}

\section{Conclusion}

\begin{frame}{Summary}

  Get the source of this theme and the demo presentation from

  \begin{center}\url{github.com/matze/mtheme}\end{center}

  The theme \emph{itself} is licensed under a
  \href{http://creativecommons.org/licenses/by-sa/4.0/}{Creative Commons
  Attribution-ShareAlike 4.0 International License}.

  \begin{center}\ccbysa\end{center}

\end{frame}

\plain{}{Questions?}

\end{document}
